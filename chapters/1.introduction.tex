\chapter{Introduction}\label{chap:introduction}

\section{Motivation}\label{sec:intro-motivation}
-- Why? Why are we here?
-- Narrative of the problem
-- Real-world problem (running injuries), impact/costs, current practical gap.
-- My personal interest in the topic?
-- Lack of conclusive research, lack of variety of techniques, etc...

\section{Context and Collaboration}\label{sec:intro-context}
-- Collaboration with Javier, ...
-- Source paper: what it proposes, what you adopt, what you don't.
    -- The value that the paper brings to the community amnd why it is important + effect on community.
    -- What could be improved?
    -- Limitations, assumptions, etc..

\section{Problem Statement \& Hypotheses}\label{sec:intro-problem-hypotheses}
-- clear and verifiable
-- Detection of running injury from kinematic data. (Rephrase better)


\section{Objectives an Scope}\label{sec:intro-objectives-scope}
-- concise, measurable, and testable
-- Scope, limits, and assumptions
-- What does the source study not cover? What is out of scope for us?
    - Longitudinal analysis
    - Follow-up on subjects to track evolution of the injury.
    - Temporal information about the injury (chronic, not chronic, recent, coping, etc...)
-- Deep Learning over kinematic data better results than classical feature-based approaches.
-- Explainability of the models -> Attempt to interpret what helps detect the injury.


\section{Contributions}\label{sec:intro-contributions}
-- What is the value of the research?
-- What assets do I leave behind?
-- Methods, data, code, results, ....

\section{Document Structure}\label{sec:intro-structure}
-- Overview of the document structure, chapters, sections, etc...
