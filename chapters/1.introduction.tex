\chapter{Introduction}\label{chap:introduction}
%Motion analysis is the systematic study of human movement, and it is a key component of many fields, including biomecanics, medicine and sports science.
%Kinematics is the study of the motion of objects without considering the forces causing the motion.
-- To populate after objectives.

Can we detect running injuries using kinematic data? -> Lots of research exists that relates injuries to training load, physiological factors that require logitudinal data to be collected during a long period of time

Lots of research into injury risk prediction, but not much into injury detection.
% Research in the domain of ML for injury risk prediction in sports has grown steadily over the last 10 years (https://bjsm.bmj.com/content/59/7/491)


Proof that Machine learning has been used successfully in the field of biomechanics 


\section{Motivation}\label{sec:intro-motivation}
-- Why? Why are we here?
-- Narrative of the problem
-- Real-world problem (running injuries), impact/costs, current practical gap.
-- My personal interest in the topic?
-- Lack of conclusive research, lack of variety of techniques, etc...

\section{Context and Collaboration}\label{sec:intro-context}
-- Collaboration with Javier, ...
-- Source paper: what it proposes, what you adopt, what you don't.
    -- What data is available, what are we taking?


\section{Objectives an Scope}\label{sec:intro-objectives-scope}
% -- concise, measurable, and testable
% -- Scope, limits, and assumptions
% -- What does the source study not cover? What is out of scope for us?
%     - Longitudinal analysis
%     - Follow-up on subjects to track evolution of the injury.
%     - Temporal information about the injury (chronic, not chronic, recent, coping, etc...)
Objectives:\\

% O1: The primary objective is to evaluate 4 classical ML models trained on the biomechanical features provided in the running-injury dataset (N=1402 runners) to classify injury status (injured vs. non-injured), and determine whether any achieves AUC-ROC > 0.7.\\

O1: The primary objective is to evaluate 4 Machine Learning models trained on engineered biomechanical features for binary injury status classification (injured vs. non-injured), and determine whether any achieve AUC-ROC > 0.7.\\

O2: Assess whether advanced deep learning models (Unilateral LSTM, Bilateral LSTM, MC-DCNN) trained on time-series kinematic data significantly outperforms the best baseline by $\Delta$AUC-ROC > 0.05 (95\% CI, p $\le$ 0.05).\\

O3: Apply post-hoc interpretability techniques (SHAP and saliency maps) to identify key kinematic features that contribute most to injury detection.

\section{Work Plan}\label{sec:intro-work-plan}
- Data acquisition and Preparation
    - Source paper data
    - Preprocessing in matlab
    - Data extraction into JSONs, folders, etc... Errors
    - Implementation of library to work with the data, both sides.
    - Data quality and checks -> We keep X side of the story. What sessions have been discarded during data acquisition?

- Exploratory Data Analysis
    - Data Exploration and visualisation; Distributions, magnitudes, ts visualisation
    - Curation of tabular and timeseries data:
        - Outlier detection and removal
    - Feature engineering:
        - TimeSeries
            - Segmentation into stance phase and normalisation
        - Tabular:
            - Study of natural clustering t-SNE and PCA.
            - Statistical tests - Study of correlation, mutual information, t-test and chisquared
            - Assumption: Lots of very related features: Collinearity analysis
            - extract Dominant Leg features
    -Feature Selection:
        - MRMR to extract top 40 features
        - Select a representant Curve for all the segments of a trial: Curve Registration and Mean Curve.

- Machine Learning modelling
    - Establish evaluation process: performance metrics, threshold selection process
    - Train classical ML models on handcrafted features
    - Hyper parameter tuning, group aware CV.
    - Analyse the results.

- Advanced Modelling:
    - Implementation of Deep Learning models: Unilateral CNN-LSTM, Billateral-CNN-LSTM, Multimodal-Billateral-CNN-LSTM and MC-DCNN
    - Hyper parameter tuning for models
    - Analyse the results

- Compare comparision:
    - Compare the models using DeLongs test on the primary evaluation metric (ROC_AUC)
    - Analyse the results and discuss the outcomes

- Model interpretation:
    - Calculate SHAP values for the best performing model of both categories.
    - Generate Saliency maps for the best performing Deep Learning model
    - Analyse SHAP values and Saliency maps to extract conclusive insights.
        - What features are most important when detecting injury?
        - What part of the stance phase is more important when detecting injury?

- Conclusions:
    - Summarise key findings.
    - Propose improvements and future lines of work.

\section{Document Structure}\label{sec:intro-structure}
-- Overview of the document structure, chapters, sections, etc...


