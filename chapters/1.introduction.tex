\chapter{Introduction}\label{chap:introduction}
% %Motion analysis is the systematic study of human movement, and it is a key component of many fields, including biomecanics, medicine and sports science.
% %Kinematics is the study of the motion of objects without considering the forces causing the motion.
% -- To populate after objectives.

% Can we detect running injuries using kinematic data? -> Lots of research exists that relates injuries to training load, physiological factors that require logitudinal data to be collected during a long period of time

% Lots of research into injury risk prediction, but not much into injury detection.
% % Research in the domain of ML for injury risk prediction in sports has grown steadily over the last 10 years (https://bjsm.bmj.com/content/59/7/491)


Running is one of the most popular forms of physical activity worldwide. However, it is also associated with a high incidence of injuries, which are common among both recreational and competitive runners \citep{vanGent2007}. These injuries often lead to pain, reduced performance, and time away from training or competition.

Early detection of running-related injuries is critical \citep{Mamom2023}. Timely identification of injury onset can enable medical intervention before a minor condition progresses into a chronic or even career-threatening disability. In this sense, accurate and early detection methods could substantially improve rehabilitation outcomes and reduce the long-term burden of injuries on athletes.  

While considerable research has focused on injury prevention and the identification of risk factors, injury detection itself remains an underexplored area in comparison. Previous research exists \citep{Mamom2023}, but compared to prevention-oriented studies, detection-oriented approaches have received significantly less attention, despite their potential for direct clinical and performance impact. Bridging this gap represents an opportunity for research with both scientific and societal relevance.  

% Do i have papaers to support "or even contradictory associations"?
One of the central challenges in this domain is the inconsistency of findings regarding the relationship between kinematic data and injury status. Previous studies have reported inconclusive associations, making it difficult to establish universal biomechanical markers of injury \citep{Peterson2022, Malisoux2024, Guan2022}. This lack of consensus highlights the need for state-of-the art analytical approaches that can integrate complex biomechanical signals and reveal hidden patterns.  

Recent advances in marker-based motion capture and machine learning have made it feasible to analyse large-scale gait datasets, offering new opportunities for detailed biomechanical analysis \citep{Harris2022}. At the same time, machine learning techniques have demonstrated strong potential in extracting meaningful patterns from such datasets, making them valuable tools for injury-related research \citep{Kokkotis2022}. In particular, deep learning methods have shown promising results in related applications, suggesting their suitability for detecting injuries directly from time-series kinematic data \citep{Sadr2025}.  


\section{Motivation}\label{sec:intro-motivation}

For these reasons, this thesis investigates whether running injuries can be detected effectively using both classical machine learning and deep learning models applied to kinematic and biometric data. In addition, the use of model interpretability techniques is explored to enhance transparency and provide biomechanical insights, ensuring that the findings contribute not only to predictive performance but also to understanding the underlying mechanisms of injury.

This project also stems from the author's interest, both as a runner and as a computer scientist, in combining biomechanics and artificial intelligence to address a tangible real-world challenge where explainability adds practical value.

\section{Context and Collaboration}\label{sec:intro-context}
This work was carried out in collaboration with Dr. Javier Gámez Payá, professor at the Faculty of Physiotherapy of the European University of Valencia (UEV), and builds upon the publicly available dataset released by \citet{Ferber2024}. The source paper provides a comprehensive pipeline for recording and preprocessing treadmill running and walking sessions for 1798 subjects. We adopt the dataset and MATLAB preprocessing routines but develop a Python-based workflow for data extraction, analysis, and modelling tailored to injury detection. Only running sessions are considered, and the input dataset is limited to kinematic, descriptive biomechanical and biometric variables.

\section{Objectives an Scope}\label{sec:intro-objectives-scope}
% Objectives:\\

% % O1: The primary objective is to evaluate 4 classical ML models trained on the biomechanical features provided in the running-injury dataset (N=1402 runners) to classify injury status (injured vs. non-injured), and determine whether any achieves AUC-ROC > 0.7.\\

% O1: The primary objective is to evaluate 4 Machine Learning models trained on engineered biomechanical features for binary injury status classification (injured vs. non-injured), and determine whether any achieve AUC-ROC > 0.7.\\

% O2: Assess whether advanced deep learning models (Unilateral LSTM, Bilateral LSTM, MC-DCNN) trained on time-series kinematic data significantly outperforms the best baseline by $\Delta$AUC-ROC > 0.05 (95\% CI, p $\le$ 0.05).\\

% O3: Apply post-hoc interpretability techniques (SHAP and saliency maps) to identify key kinematic features that contribute most to injury detection.
The project pursues the following objectives:
\begin{itemize}
    \item[O1] Evaluate classical machine-learning models trained on engineered biomechanical features for binary injury status classification (injured vs. non-injured) and determine whether any achieve AUC-ROC $> 0.7$.
    \item[O2] Assess whether advanced deep learning models trained on time-series kinematic data outperform the best baseline by $\Delta$AUC-ROC $> 0.05$ (95\% CI, $p \le 0.05$).
    \item[O3] Apply explainability techniques to identify key kinematic features that contribute most to injury detection.
\end{itemize}
The scope of the study is restricted to binary injury status detection. It does not attempt to model injury severity, specific injury types, or long-term risk.
% TODO: Add a business description.

\section{Work Plan}\label{sec:intro-work-plan}
The project is structured in three phases summarised below.
\subsection{Phase 1: Data preparation and exploratory analysis}
\begin{itemize}
    \item \textbf{Data acquisition and preprocessing}
    \begin{itemize}
        \item Ingest source paper data and reproduce the MATLAB preprocessing pipeline.
        \item Export structured artifacts (e.g., JSON, CSV, organized folders) and document any errors or deviations.
        \item Implement a Python library to load, transform, and validate both tabular and time-series data.
        \item Perform data quality checks and record excluded or discarded sessions with justification.
    \end{itemize}
    \item \textbf{Exploratory Data Analysis (EDA)}
    \begin{itemize}
        \item Visualize distributions, magnitudes, and representative time-series.
        \item Curate tabular and time-series data; detect and remove outliers.
        \item Feature engineering:
        \begin{itemize}
            \item Time-series: segment stance phase and normalize trajectories.
            \item Tabular: explore natural structure (t-SNE, PCA); run correlation, mutual information, t-test, and chi-squared tests; assess collinearity; derive dominant-leg features.
        \end{itemize}
        \item Feature selection: apply MRMR to identify a compact set of top features (e.g., top 40).
        \item Curve summarization: perform curve registration and compute mean representative curves for trial segments.
    \end{itemize}
\end{itemize}

\subsection{Phase 2: Modeling and comparative evaluation}
\begin{itemize}
    \item \textbf{Evaluation protocol}
    \begin{itemize}
        \item Define primary and secondary metrics, with emphasis on AUC-ROC and threshold-selection strategy.
        \item Use group-aware cross-validation for fair subject-level evaluation and reproducible hyper-parameter tuning.
    \end{itemize}
    \item \textbf{Classical machine learning (handcrafted features)}
    \begin{itemize}
        \item Train baseline models on engineered features.
        \item Tune hyper-parameters with group-aware CV and analyze performance.
    \end{itemize}
    \item \textbf{Advanced deep learning (time-series kinematics)}
    \begin{itemize}
        \item Implement and train Unilateral and Bilateral CNN--LSTM architectures and an MC-DCNN model.
        \item Tune model capacity and regularization; monitor learning dynamics and generalization.
    \end{itemize}
    \item \textbf{Comparative analysis}
    \begin{itemize}
        \item Compare best classical and deep models using DeLong's test on the primary metric (AUC-ROC).
        \item Summarize findings and discuss practical implications.
    \end{itemize}
\end{itemize}

\subsection{Phase 3: Model interpretation}
\begin{itemize}
    \item Compute SHAP values for the best-performing classical model to identify influential features.
    \item Generate saliency maps for the best-performing deep model to localize informative regions across the stance phase.
    \item Synthesize insights to address: (i) which features contribute most to injury detection, and (ii) which portions of the stance phase are most informative.
\end{itemize}

\section{Document Structure}\label{sec:intro-structure}
This thesis is organised in five chapters. The first chapter sets the context and frames the injury detection problem. Chapter~\ref{chap:state-of-the-art} summarises the current state of the field of Machine learning applied to biomechanics and highlights the challenges that motivate this study. Chapter~\ref{chap:materials-methods} presents the dataset of \citet{Ferber2024}, describes the preprocessing pipeline, and details the modelling and evaluation procedures. Chapter~\ref{chap:results-discussion} reports and discusses the experimental results, including model interpretability analyses. Finally, Chapter~\ref{chap:conclusions} summarises the findings, outlines limitations, and proposes directions for future research.
