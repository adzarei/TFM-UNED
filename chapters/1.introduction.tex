\chapter{Introduction}\label{chap:introduction}
%Motion analysis is the systematic study of human movement, and it is a key component of many fields, including biomecanics, medicine and sports science.
%Kinematics is the study of the motion of objects without considering the forces causing the motion.
-- To populate after objectives.

Can we detect running injuries using kinematic data? -> Lots of research exists that relates injuries to training load, physiological factors that require logitudinal data to be collected during a long period of time

Lots of research into injury risk prediction, but not much into injury detection.
% Research in the domain of ML for injury risk prediction in sports has grown steadily over the last 10 years (https://bjsm.bmj.com/content/59/7/491)


Proof that MAchine learning has been used successfully in the field of biomechanics 




\section{Motivation}\label{sec:intro-motivation}
-- Why? Why are we here?
-- Narrative of the problem
-- Real-world problem (running injuries), impact/costs, current practical gap.
-- My personal interest in the topic?
-- Lack of conclusive research, lack of variety of techniques, etc...

\section{Context and Collaboration}\label{sec:intro-context}
-- Collaboration with Javier, ...
-- Source paper: what it proposes, what you adopt, what you don't.
    -- The value that the paper brings to the community amnd why it is important + effect on community.
    -- What could be improved?
    -- Limitations, assumptions, etc..

\section{Problem Statement \& Hypotheses}\label{sec:intro-problem-hypotheses}
-- Clear and verifiable
-- Detection of running injury from kinematic data. (Rephrase better)


\section{Objectives an Scope}\label{sec:intro-objectives-scope}
% -- concise, measurable, and testable
% -- Scope, limits, and assumptions
% -- What does the source study not cover? What is out of scope for us?
%     - Longitudinal analysis
%     - Follow-up on subjects to track evolution of the injury.
%     - Temporal information about the injury (chronic, not chronic, recent, coping, etc...)
Objectives:\\

O1: The primary objective is to evaluate 4 classical ML models trained on the biomechanical features provided in the running-injury dataset (N=1402 runners) to classify injury status (injured vs. non-injured), and determine whether any achieves AUC-ROC > 0.7.\\

O2: Assess whether advanced deep learning models (Unilateral LSTM, Bilateral LSTM, MC-DCNN) trained on time-series kinematic data significantly outperforms the best baseline by $\Delta$AUC-ROC > 0.05 (95\% CI, p $\le$ 0.05).\\

O3: Apply post-hoc interpretability techniques (SHAP and saliency maps) to identify key kinematic features that contribute most to injury detection.

\section{Contributions}\label{sec:intro-contributions}
-- What is the value of the research?
-- What assets do I leave behind?
-- Methods, data, code, results, ....

\section{Document Structure}\label{sec:intro-structure}
-- Overview of the document structure, chapters, sections, etc...
