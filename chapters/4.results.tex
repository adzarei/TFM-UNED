\chapter{Results and Discussion}\label{chap:results-discussion}
-- Comback to objectives and hypotheses.
-- Comparison with SoTA and Source Paper.
-- limitations
-- practical implications.

\section{Baseline Model}\label{sec:results-baselines}

\section{Dominant Leg Model}\label{sec:results-dominant-leg}  % Rename....

\section{Deep Learning models}\label{sec:results-deep-learning}



\section{Comparison}\label{sec:results-comparison}
... Accross the CV, the best advanced model achieved AUC-ROC 0.54 (95\% CI) %TODO: Accross CV folds
, compared with the best baseline (0.76). The DeLong's test yielded p = 0.xx, indicating no statistical significant improvement.

The lack of Deep Learning improvement suggests the dataset may be too small or too noisy for more complex models to help (intra-session variability). Strong handcrafted features (peaks, etc..) likely already capture most of the useful information. In addition, results are strongly influenced by who the runner is due to variability between runners, which makes it harder to generalise across people.

\section{Interpretability}\label{sec:results-interpretability}
-- Saliencey maps, feature importance, ...
-- Can we explain the results?
-- Can we say what data is more important?

% With the amount of data we have, deep learning suffers from the lack of data.
% Removing one side of the data seems to remove some udeful information which suggests that further exploratio nin the assymetries of the gait.
% Think how cna I help the next person so that he does not find the same problems.
% Focus on provding information for the community


\section{Discussions} % Should it be a separate chapter.
% To discuss what the directions to explore after my work:
% Geetting more data
% focusing on the assymetry
% Summarise all of the segment into one curve rrpresentative loosing data.
% ....
% Intra subject variability
% Very Non-linear relationship, probably our models are too simple
% ...
% Biomechanics community sharping the ir code i.e. the paper that got good DeepLearning resutls from the same data. I f I could replicate, I could build on top and get better results. Otherwise too generic to replicate.
