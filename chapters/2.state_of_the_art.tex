\chapter{State of the Art}\label{chap:state-of-the-art}
-- What was done before?
-- Proof of research

-- Previous work with Classical ML models
-- Previous work with Deep Learning models
-- Previous work with Same source paper
-- Previous work in Injury detection

\section{Running Biomechanics}\label{sec:sota-biomechanics}
-- Signals (angles, angular velocities), markers, stride/stance cycles.
-- Kinematics, Kinetics, etc...
-- Reference a Paper from Javier if possible. + The paper he shared.
-- Previous work with Kinematic data
-- Cycle Segmentation and normalization to 101 points
    -- Challenges with TS variability

\cite{Napier2019} conducted a comprehensive analysis of kinematic correlates of kinetic outcomes associated with running-related injury in healthy novice female runners. They found that 46\% of the variance in vertical impact transient could be explained by kinematic variables, with speed, horizontal distance from heel to center of mass, foot strike angle, and step length being the most significant predictors. Notably, hip flexion angle and shank angle did not contribute significantly to the predictive models.


\citet{Vannatta2023} Asymmetries are context specific: asymmetry can be expected between limbs during running that may pose challenges to informing clinical decisions. Methods used to quantify asymmetry appear to offer a simple way to make relative comparisons between limbs during running. However, if one is seeking to employ cut-off values, consideration should be given to the asymmetry method used, the joint assessed, and the variable being measured.

\cite{Malisoux2024} Gait asymmetry was not associated with higher injury risk for investigated spatiotemporal and kinetic variables. No between-limb differences were observed in runners having sustained an injury.



\section{Data Science in Biomechanics}\label{sec:sota-data-science}
-- Data collection, preprocessing, feature engineering, model training, evaluation, etc...
-- Present techniques used / considered, but no need to say why choosen or not.
-- Previous work with Classical ML models
-- Previous work with Deep Learning models

\section{Machine Learning for Injury Detection}\label{sec:sota-ml-injury-detection}
-- What data is used? Kinematics, Kinetics, etc...
-- Classical feature-based approaches, deep learning models.
-- Tabular, Timeseries data. Marker data, GRF, kinematics,....
-- PRevious work specific for the injury detection

\cite{Harris2022} Machine Learning in biomechanics. What techniques are used and results. Also motivaiton and impact.
\cite{Dibbern2025} Explainable AI and SHAP values....

\section{Source Paper}\label{sec:sota-source-paper}
% Critical Review
This projects build uppon the work done in \citet{Ferber2024}. This paper addressed the lack of benchmark datasets for quantitative biomechanical gait analysis by providing a dataset of 1798 healthy and injured subjects.
-- The value that the paper brings to the community amnd why it is important + effect on community.
-- What could be improved?
-- Better clarity on injury definition
-- Consistency on markers (not all subjects have the same marker data.)
-- Clean dataset and drop noise sessions
-- 
-- Limitations, assumptions, etc...

\section{Takeaways}\label{sec:sota-methodological-takeaways}
-- short decisions your methodology inherits from the SoTA.
-- Models, techniques, normalization, etc...
-- Why these tequeniques have been choosen?
