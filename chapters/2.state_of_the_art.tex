\chapter{State of the Art}\label{chap:state-of-the-art}
-- What was done before?
-- Proof of research

\section{Running Biomechanics}\label{sec:sota-biomechanics}
-- Signals (angles, angular velocities), markers, stride/stance cycles.
-- Kinematics, Kinetics, etc...
-- Reference a Paper from Javier if possible. + The paper he shared.

\section{Data Science in Biomechanics}\label{sec:sota-data-science}
-- Data collection, preprocessing, feature engineering, model training, evaluation, etc...
-- Present techniques used / considered, but no need to say why choosen or not.

\section{Machine Learning for Injury Detection}\label{sec:sota-ml-injury-detection}
-- What data is used? Kinematics, Kinetics, etc...
-- Classical feature-based approaches, deep learning models.
-- Tabular, Timeseries data. Marker data, GRF, kinematics,....

% Moved to introduction...
% \section{Source Paper (Critical Review)}\label{sec:sota-source-paper}
% -- The value that the paper brings to the community amnd why it is important + effect on community.
% -- What could be improved?
% -- Limitations, assumptions, etc...

\section{Takeaways}\label{sec:sota-methodological-takeaways}
-- short decisions your methodology inherits from the SoTA.
-- Models, techniques, normalization, etc...
-- Why these tequeniques have been choosen?
