\chapter{Conclusions}\label{chap:conclusions}

\section{Conclusions}\label{sec:conc-conclusions}
-- Linked to abstract, very focused.
... baseline models remain a robust choice for injury status classification on this dataset.

\section{Limitations}\label{sec:conc-limitations}

\subsection{Data limitations (sample, biases, missingness)}\label{sec:limitations-data}

\subsection{Methodological limitations (assumptions, generalization)}\label{sec:limitations-methods}
- Validation to Test score drop (0.1) in deep learning models suggests that validation guided model selection and the test set is truly unseen. An easy improvement is to use CV-based tuning exclusively with the traning set.

\subsection{Evaluation limitations (metrics, external validation)}\label{sec:limitations-evaluation}
-- Approaching the problem as a prediction of the mayority class -> Class imbalnce made things harder
    -- In the future, change to minority class

\section{Future Work}\label{sec:conc-future-work}
-- All of the work that we did not have time to do.
-- What we would do if we had more time.
-- What we would do if we had more data.
-- What we would do if we had more compute.
-- Look into other parts of the gait cycle? Or use the full cycle isntead of only the stance.
-- curver summarization in 3D, not only 1D to preserve better the shape.
-- Maybe try the propblem in 2D?
-- analysing the saliency maps by injury type can help undersantd difference between injuries. for example: Knee injuries will probably havea difference map that fascitic plantar, because the respective parts of the body are exercises and strained during different phases.
-- Run permutation importance on deeplearning models, but making sure permutatiosn are done channel-wise.
