\chapter{Materials and Methods}\label{chap:materials-methods}
This chapter introduces the materials and datasets provided in the source paper \citep{Ferber2024}. We then describe how the data were preprocessed and explored, explain how the models were implemented, trained, and evaluated, and finally present the methods used to analyse model explainability.

\section{Data and Materials}\label{sec:method-data-materials}
As part of \citet{Ferber2024}, the authors have published a dataset of 1798 \glspl{subject} running or walking on a treadmill captured using multi-camera 3D motion capture (\gls{mocapsys}) equipment. The dataset is composed of raw 3d marker data of each \gls{session}, descriptive biomechanical variables derived from the marker data for each session and demographic and antropometric metadata of the subjects. Additionally, the matlab code for the preprocessing of the data, the calculation of the kinematic variables and tutorial notebooks that help ilustrate how to use their library and read the data from the folder structure are also provided. Although both walking and running data is provided, we have used exclusively running data. From this point onwards, walking data is ignored for brevity.

\subsection{Measurement protocol}\label{subsec:measurement-protocol}
The complete measurement protocol is described in \citet{Ferber2024}. In this subsection, we provide a summary of the aspects that are most relevant for the methods and analyses presented in this thesis.

The raw marker data was collected using high-speed optoelectronic infrared-based motion capture cameras (MX3/Bonita, Vicon, Oxford, UK) were used to record the position of 9mm spherical retro-reflective markers attached to anatomical landmarks of the subjects at either 120Hz or 200Hz. Depending on the lab, eight or three cameras were used. Below we list and describe the three sets of markers that were used.
\begin{itemize}
    \item \textbf{Core}: Markers attached to the following anatomical landmarks: medial and lateral malleoli, medial and lateral femoral condyles, and greater trochanters.
    \item \textbf{Additional}: Markers attached to the following anatomical landmarks: bilateral 1st and 5th metatarsal heads, distal aspect of the shoe, tibial tuberosity, anterior superior iliac spines, and iliac crests;
    \item \textbf{Clusters}: 3 or 4 markers placed on rigid shells attached to the following anatomical landmarks: sacrum, bilateral thigh and shank, and posterior aspect of both shoes.
\end{itemize}

During the recording of the sessions of all 1798 subjects the 'core' and 'cluster' set of markers were used. The additional set was only used during the recording of the sessions of 1082 subjects. Figure \ref{fig:marker_position} shows the lower body of a subject and the three sets of markers.

\begin{figure}[ht]
    \begin{centering}
    \includegraphics[width=0.5\columnwidth]{images/billateral_marker_position.png}
    \par\end{centering}
    \caption{Position of markers on the subjects. Grey markers correspond to core set of markers set, black to additional set and white to the clusters set.}
    \label{fig:marker_position}
\end{figure}

% TODO: Explain the recording procedure: Warm up, duration, discard, etc...

\subsection{Dataset Description}\label{subsec:method-dataset-description}
As described above, the dataset contains two complementary components: (i) session-level metadata stored as tabular files, and (ii) motion-capture marker data together with derived kinematic variables per session. Below we describe each component and list the variables that are available for analysis.

\paragraph{Metadata (tabular)} Metada is provided as a CSV file: \texttt{run\_data\_meta.csv}. Each row corresponds to a single treadmill session recorded for a subject.
\begin{itemize}
    \item \texttt{sub\_id}: Unique subject identifier.
    \item \texttt{datestring}: Session recording date.
    \item \texttt{filename}: Session filename used to locate the marker json file for the session.
    \item \texttt{speed\_r}: Treadmill belt speed (m/s).
    \item \texttt{age}: Subject's age (years).
    \item \texttt{Height}: Body height (cm).
    \item \texttt{Weight}: Body weight (kg).
    \item \texttt{Gender}: Subject gender.
    \item \texttt{DominantLeg}: Dominant leg (Left/Right/Ambidextrous).
    \item \texttt{InjDefn}: Injury severity reported by the subject choosing between 4 options: (1) No Injury, (2) Continuing to train in pain, (3) Training volume/intensity affected, (4) Minimum of two workouts missed in a row.
    \item \texttt{InjJoint}, \texttt{InjSide}, \texttt{SpecInjury}, \texttt{InjDuration}: Primary injured joint, side (Left/Right), Specific injury diagnosis from medical professional, and for how long has the subject had the injury.
    \item \texttt{InjJoint2}, \texttt{InjSide2}, \texttt{SpecInjury2}: Secondary injury information (if applicable).
    \item \texttt{Activities}: Subject reported athletic activities performed on a regular basis.
    \item \texttt{Level}: Self-reported level of athletic activity (recreational/competitive).
    \item \texttt{YrsRunning}: Number of years subject has been running on a regular basis.
    \item \texttt{RaceDistance}: Perferred race distance.
    \item \texttt{RaceTimeHrs}, \texttt{RaceTimeMins}, \texttt{RaceTimeSecs}: Perferred race distance best time.
    \item \texttt{YrPR}: Year of preferred race distance personal best time.
    \item \texttt{NumRaces}: Number of races completed per year.
\end{itemize}

\paragraph{Marker and descriptive biomechanical data (Hierarchical)} For each session, 3D positions of reflective markers and descriptive variables derived from those markers are provided.
\begin{itemize}
    \item \textbf{Recording frequency} Fequency of the recording in Herz (Hz). Either 120hz or 200hz depending on the lab.
    \item \textbf{Joint Marker Neutral}: 3D coordinates of the individual markers of the core and additional marker sets while the subject is standing in neutral position.
    \item \textbf{Cluster Marker Neutral}: 3D coordinates of the individual markers in the cluster marker set while the subject is standing in neutral position.
    \item \textbf{Dynamic Marker data}: 3D coordinates at each of the frames of the recording for the individual markers in the cluster marker set.
    \item \textbf{Descriptive biomechanical variables}: Set of 77 variables derived from the 3D coordinates Marker data. Calculated for Left and Right side of the body.
\end{itemize}

It must be noted that \texttt{Joints} and \texttt{Neutral} coordinates do not represent the true anatomical centers, only the centers of the markers on the skin of the subject.
Only 33 of the 77 descriptive biomehanical variables are populated per side, the rest all contain 0 for all values of all sessions, this totals 66 populated descriptive variables in total. Table \ref{tab:met-desc-variable-init} describes the variable set per side.

\begin{table}[ht]
    \centering
    \caption[Populated Descriptive Variables Summary]{Summary of the Descriptive Biomechanical Variables that are populated\label{tab:met-desc-variable-init}}
    \begin{tabular}{lp{0.6\textwidth}}
    \hline
    Variable & Category \\
    \hline
    Step width (m) & Temporal-spatial \\
    Stride rate (steps/min) & Temporal-spatial \\
    Stride length (m) & Temporal-spatial \\
    Swing time & Temporal-spatial \\
    Stance time & Temporal-spatial \\
    Peak drop angle & Pelvis kinematics \\
    Drop excursion & Pelvis kinematics \\
    Dorsiflexion peak angle & Ankle kinematics \\
    Ankle Eversion peak angle & Ankle kinematics \\
    Ankle Rotation peak angle & Ankle kinematics \\
    Ankle Eversion excursion & Ankle kinematics \\
    Ankle Rotation excursion & Ankle kinematics \\
    Ankle Eversion \% of stance & Ankle kinematics \\
    Knee Flexion peak angle & Knee kinematics \\
    Knee Adduction/abduction peak angle & Knee kinematics \\
    Knee Rotation peak angle & Knee kinematics \\
    Knee Adduction/abduction excursion & Knee kinematics \\
    Knee Rotation excursion & Knee kinematics \\
    Hip Extension peak angle & Hip kinematics \\
    Hip Adduction peak angle & Hip kinematics \\
    Hip Rotation peak angle & Hip kinematics \\
    Hip Adduction excursion & Hip kinematics \\
    Hip Rotation excursion & Hip kinematics \\
    Foot progression angle & Foot kinematics \\
    Heel-strike angle & Foot kinematics \\
    Medial heel whip excursion from toe-off & Foot kinematics \\
    Ankle eversion peak velocity & Joint velocities \\
    Ankle rotation peak velocity & Joint velocities \\
    Knee adduction peak velocity & Joint velocities \\
    Knee abduction peak velocity & Joint velocities \\
    Hip abduction peak velocity & Joint velocities \\
    Knee rotation peak velocity & Joint velocities \\
    Hip rotation peak velocity & Joint velocities \\
    Pelvic drop peak velocity & Joint velocities \\
    Pronation onset \% of gait cycle & Foot timing \\
    Pronation offset \% of gait cycle & Foot timing \\
    Vertical oscillation (mm) & Vertical oscillation \\
    \hline
    \end{tabular}
\end{table}

All descriptive variables follow the interpretations described in \citet{Bartlett2014}. Unless specified explicitly in the table, the units are degrees for angles and excursions, and degrees/s for joint velocities. Excursions are also commonly known as range of motion (ROM) in the literature. 
% TODO: Explain important biomechanical concepts like dorsiflexion, eversion, etc...

\subsection{Code Description}\label{subsec:method-code-description}
% -- Matlab code provided by the source paper.
% -- My code and repo (high level + refernce to Anex 1)
We have implemented an ad-hoc python library to orchestrate data extraction, preprocessing, feature engineering, modelling, and evaluation. All code developed for this project and used to produce the reported results has been made publicly available for reproducibility \cite{Zapater_Reig_Running_Injury_Clinic}. Additionally we have made use of the MATLAB reference implementation published with the dataset in \cite{Ferber2024}. The matlab scripts used have also been included in the repository in the \texttt{suplemental_material} folder which are available for consultation. Below we describe the most important scripts that we borrow from the paper:

\paragraph{\texttt{gait\_kinematics.m}} Calculates joint angles and joint angle velocities from \texttt{Dynamic Marker Data}, \texttt{Join Marker Neutral} and \textt{Cluster Marker Neutral} data. It constructs anatomical segment coordinate systems from the neutral marker data, tracks segment motion during using the dynamic marker data and computes XYZ Cardan joint angles and angular velocities. It returns:
\begin{itemize}
    \item \texttt{Angles}: per-frame joint/segment angles for L/R ankle, knee, hip; foot and pelvis segments (degrees).
    \item \texttt{Velocities}: per-frame angular velocities for the same structures (deg/s).
    \item \texttt{jc}: Estimated joint centre positions (pelvis, hips, knees, ankles).
\end{itemize}

\paragraph{\texttt{gait\_steps.m}} Calculates the descriptive biomechanical variable set from \texttt{Dynamic Marker Data}, \textt{Cluster Marker Neutral} data, \textt{Angles} and \textt{Velocities}. It returns:
\begin{itemize}
    \item \texttt{event}: Gait cycle events by side: Touchdown, midstance, toe-off and heelwhip.
    \item \texttt{Descriptive biomechanical variables}: Set of variables described in table \ref{tab:met-desc-variable-init}
\end{itemize}

Both script output additional variables, but are not of relevance for this project.

\subsection{Kinematic Data Extraction}\label{subsec:kinematic-data-extraction}
-- How Kinematic variables are derived.
-- How we obtained angles, anglular velocities from matlab and events...
-- Internal processing assumptions
-- Axis reconstruction
-- Estimations
-- Focus on the Stance because computed variables are also focused there. Same reference paper. Or alternatively put in state of the art and justify here that we think it is better this way.

% TODO: How to integrate EDA and Preprocessing, since it is realted...


\section{Exploratory Data Analysis (EDA)}\label{sec:method-eda}
-- Exploration → Extraction → Preparation, with concrete checks and figures:
-- Tabular data Exploration (sanity and distributions)
-- Dataset inventory (sessions per runner, class prevalence per runner, cycles/session).
-- Missingness map (metadata \& signals); strategy preview (drop/impute).
-- Outliers: per-joint angle/velocity ranges, z-score >3 heatmap by runner (flag sensors/markers).
-- Dimensionality previews: PCA/t-SNE to see separability.


-- Timeseries data Exploration:
-- Visuals: overlay of 20 random stance cycles per class; Markers, Angles, Velocities, ...
-- Analysis of curves
-- Extraction (curve-level descriptors)


-- Preparation (decisions fixed before modelling)

-- Group-aware split policy (runner-level), ensuring no cycle leakage across folds.
-- Class-imbalance diagnostics (AUC-PR baseline, per-runner prevalence plots).
-- Filtering rules for cycles/sessions (min cycles per session, quality thresholds).

-- Final feature set(s) to carry forward (e.g., summaries, PC scores, raw curves for DL)

% TODO: Maybe we keep outliers and filering here?


\section{Pre-processing}\label{sec:method-preprocessing}
-- Tabular:
    -- filtering, subject identification, outliers, cleaning, etc..
    -- Final dataset creation -> Give name to track

-- Timeseries:
    -- Extraction of angles and angular velocities as mentioned in literature.
    -- Extraction of events
    -- Combination of events and TS
    -- Cycle Segmentation and normalization -> Reference papers either here or in SOTA.

\subsection{Feature Engineering}\label{subsec:method-feature-engineering}
-- Transformations and feature extraction
    -- Dominant Leg -> VIF reduction
    -- Obtaining a representative curve
        -- Curve registration attempt and Actual result

-- Feature selection:
    -- MRMR for feature selection

\subsection{Feature Selection}\label{subsec:method-feature-selection}
-- Transformations and feature extraction
    -- Dominant Leg -> VIF reduction
    -- Obtaining a representative curve
        -- Curve registration attempt and Actual result

-- Feature selection:
    -- MRMR for feature selection

% \subsection{Tabular Data}\label{subsec:method-tabular-data}
% -- Correlations mutual information, ...
% -- Collinearity: PCA, VIF, ...
% -- Left-right features - Dominant leg extraction
% -- Collinearity reduction

% \subsection{Timeseries Data}\label{subsec:method-timeseries-data}
% -- Segmentation: TD/TO, stance, swing, ...
% -- Normalisation/standardisation

\subsection{Final Feature Sets}\label{subsec:method-final-feature-sets}
-- Summary of the feature by dataset.

\section{Models}\label{sec:method-models}
\subsection{Baseline (Tabular)}\label{subsec:method-baselines}
\subsection{Deep Learning}\label{subsec:method-deep-learning}
-- Unilateral LSTM
-- Bilateral LSTM
-- Multimodal
-- TMAG
-- MC DCNN

-- Add special tuning for class imbalance, etc...
-- Architecture choices (hyperparameters, etc...)
-- class imbalance, oversampling, etc... TBD Where to put this? -> Explain in the models that used it. We will give results for each option individually.

\section{Evaluation Protocol}\label{sec:method-evaluation-protocol}
-- The blueprint before we run anything.
-- Questions/Hipothesis, grouping, leakage, validation scheme (group-aware train/val/test)
-- Evaluation protocol -> Measurements, statistics. How are we scoring the experimentriment. (Primary, Secondary metric, Threshold rule, ...).
-- Significance Tests. How do we know something is "Better" (DeLong's test)?
-- CI 95 \%
-- Research questions (RQs), validation scheme (group-aware train/val/test), splits, primary/secondary metrics.
-- Justify the choices , but do not present any results yet...

\subsection{Evaluation Metrics}\label{subsec:method-evaluation-metrics}
- AUC ROC + PR for class imbalance + Macro F1 score for threshold.

\subsection{Training and Validation Strategy}\label{subsec:method-training-validation-strategy}
-- Train, test, validation split.
-- Group-aware split policy (runner-level), ensuring no cycle leakage across folds.
-- Class-imbalance diagnostics (AUC-PR baseline, prevalence).

\subsection{Hyperparameter Tuning}\label{subsec:method-hyperparameter-tuning}
-- Grid search, random search, ...
-- Cross-validation, nested cross-validation, ...
-- Early stopping, ...
-- Hyperparameter tuning strategy.


\section{Explainability}\label{sec:method-explainability}
<Draft> This project would have been set up differently if the sole goal had been to build and deploy an injury detection model. In that case, explainability would mostly be used just to debug the model, and the focus would shift toward making data collection easier, choosing an efficient architecture, and reducing resource use. In our case, though, explainability is central: a deep learning model with an AUC-ROC of 0.7 doesn't add much value if it remains a black box.

To get the most out of this work, we used both intrinsic methods and Explainable AI (XAI) techniques, as in \cite{FuentesJimnez2025}, to make the model's decisions more transparent.

\subsection{Intrinsic Explainability}\label{subsec:method-intrinsic-explainability}
- Random Forest, linear regression, etc...
-- Pros, Cons, etc..
-- what affects them?

\subsection{Post-hoc Explainability}\label{subsec:method-posthoc-explainability}
-- SHAPs, Saliency maps, feature permutation?



% \section{Reproducibility Assets}\label{sec:method-reproducibility}
% -- code repository structure, libraries, etc...
