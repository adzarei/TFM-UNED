\title{Story of the Thesis:}

Intro - Motivation - collaboration:
There is this paper that Javier shared with us, it is really interesting because XYZ
one of the biggest sources of this kind of data that exists, etc…
Interesting because healthy and injured
Lots of injured, usually not the case
Wide range of ages, etc..
Set standard for data sharing / collection (MoCap System, ref paper explaining)
Unusual: Most of the dataset is of injured people

Injury detection is a key problem to solve, because, XYZ,
Sport trending
Sport impact on people’s lives, normal and competitive
People have injury, but do not know it, impact on them, etc…
XXX

Injury detection problem itself has the following challenges:
Inter-subject and Intra-trial variability
Inter-injury variability; There is no perfect gait.
Each person has it’s own gait/posture specificities -> No link to injury
Lots of work done on prediction, but little on detection??? Check
XXX

Machine Learning has been used in the past to approach this data and problem:
Deep Learning in biomechanics
Deep Learning able to learn complex relation-ships to overcome the problems with variability
Classical ML models on Biomechanics
Paper working on the same data
Classical ML vs Deep Learning?
Explainability?


For those reasons, the objective of this thesis is to:
O1: The primary objective is to evaluate 4 classical ML models trained on the biomechanical features provided in the running-injury dataset (N=1402 runners) to classify injury status (injured vs. non-injured), and determine whether any achieves AUC-ROC > 0.7.\\

O2: Assess whether advanced deep learning models (Unilateral LSTM, Bilateral LSTM, MC-DCNN) trained on time-series kinematic data significantly outperforms the best baseline by $\Delta$AUC-ROC > 0.05 (95\% CI, p $\le$ 0.05).\\

O3: Apply post-hoc interpretability techniques (SHAP and saliency maps) to identify key kinematic features that contribute most to injury detection.


-- Limit of what an M4 can provide in a reasonable time
-- Limit to the access to papers and tools provided with the university access



