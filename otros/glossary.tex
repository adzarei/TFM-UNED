\makeglossaries

% Introduzca de cada término del glosario su nombre y su descripción
\newglossaryentry{subject}{
    name={subject},
    description={An individual runner included in the dataset. Each subject may contribute one or more running sessions.}
}

\newglossaryentry{session}{
    name={session},
    description={A continuous treadmill run (approximately 60 seconds) recorded from a single subject. Each session yields multiple gait cycles.}
}

\newglossaryentry{cycle}{
    name={cycle},
    description={A gait cycle defined as the interval between two successive foot touchdowns of the same foot, including both stance and swing phases.}
}


\newglossaryentry{mocapsys}{
    name={MoCap system},
    description={Motion Capture System. A system that uses multiple cameras and reflective markers to record the three-dimensional positions of anatomical landmarks on a subject during movement, enabling precise measurement and analysis of body kinematics.}
}

\newglossaryentry{matlab}{
    name={matlab},
    description={A high-level programming language and programming computing platform for engineering and scientific applications, including data analysis and signal processing.}
}

\newglossaryentry{jupyter}{
    name={jupyter},
    description={A web-based interactive development environment for notebooks, code, and data. It is typically used in data science, scientific computing, computational journalism, and machine learning.}
}

\newglossaryentry{channel}{
    name={channel},
    description={In time-series data, a channel refers to a single variable or measurement recorded over time. For example, each marker coordinate (e.g., left knee x-position) constitutes a separate channel. Multichannel time-series data contain multiple such variables recorded simultaneously.}
}
